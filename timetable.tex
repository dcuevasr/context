\documentclass{report}
\usepackage{graphicx}
\usepackage{breakcites}
\usepackage{natbib}
\usepackage{amsmath}
\usepackage[margin=1.5cm]{geometry}
\usepackage{tabularx}
\bibliographystyle{apalike}

% Remember to start reftex-mode

\begin{document}
\section*{Brief project description}
The idea of the project is the following: We will create an agent that learns to control an arm in environment A (say, the real world carrying nothing), and learns how to control the arm in environment B (say, with high winds). The agent then learns to recognize environments A and B and to deploy the right controller for each environment depending on this recognition.

We will argue that this is how humans do it and present evidence (based on literature research) for the predictions this model makes, as well as try to make new predictions.

After this methods paper is done, model validation can be carried out in future studies.


\section*{Time plan for the arm movement project.}

\begin{tabularx}{\textwidth}{
  l|
  >{\hsize=.4\hsize\linewidth=\hsize}X|
  >{\hsize=.2\hsize\linewidth=\hsize}X|
  >{\hsize=.2\hsize\linewidth=\hsize}X}
% {l|{\hsize=.4\hsize}X|{\hsize=0.3\hsize}X|{\hsize=0.3\hsize}X}
End Date & Milestone & A priori comments & A posteriori comments \\ \hline \hline
12.02.2021 & Learn about published models on arm-movement planning, especially those that can be used as generative models & & \\ \hline
19.02.2021 & Learn about measures for how practiced a movement is. Reaction times? Speed profiles? Trajectories? & & \\\hline
12.03.2021 & Implement the best model/s and do some testing as to its capabilities. & Without inference &   \\ \hline
19.03.2021 & Add the sequential goals as another level in the hierarchy. & Still no inference. & \\ \hline
02.04.2021 & Run some exploratory simulations & How well does the model behave? Are there bad cases? & \\ \hline
09.04.2021 & Talk a  bout possible predictions that can be made with the model. & This includes a meeting. & \\ \hline
--- & Run final simulations and write paper. & & \\ \hline

\end{tabularx}


\end{document}