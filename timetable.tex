\documentclass{report}
\usepackage{graphicx}
\usepackage{breakcites}
\usepackage{natbib}
\usepackage{amsmath}
\usepackage[margin=1.5cm]{geometry}
\usepackage{tabularx}
\bibliographystyle{apalike}

% Remember to start reftex-mode

\begin{document}
\section*{Brief project description}
In this chapter, I describe the everything related to the paper that should come
out of this project, including modeling and experimental findings, as well as
the direction of the paper and things needed to get there.

\section{Project description}
The idea of the project is the following: We will create an agent that learns to
control an arm in environment A (say, the real world carrying nothing), and
learns how to control the arm in environment B (say, with high winds). The agent
then learns to recognize environments A and B and to deploy the right controller
for each environment depending on this recognition.

We will argue that this is how humans do it and present evidence (based on
literature research) for the predictions this model makes, as well as try to
make new predictions.

After this methods paper is done, model validation can be carried out in future
studies.

\section{Planning}
Here I enumerate the things I need to learn and do in order to get this project
moving and to finish it.

\begin{tabularx}{\textwidth}{
  l|
  >{\hsize=.4\hsize\linewidth=\hsize}X|
  >{\hsize=.2\hsize\linewidth=\hsize}X|
  >{\hsize=.2\hsize\linewidth=\hsize}X}
% {l|{\hsize=.4\hsize}X|{\hsize=0.3\hsize}X|{\hsize=0.3\hsize}X}
End Date & Milestone & A priori comments & A posteriori comments \\ \hline \hline
12.02.2021 & Learn about published models on arm-movement planning, especially those that can be used as generative models & & Discovered low-hanging fruit. Shifting directions a bit. See \ref{subsec:proposal}\\ \hline
19.02.2021 & Learn about measures for how practiced a movement is. Reaction times? Speed profiles? Trajectories? & & \\\hline
12.03.2021 & Implement the best model/s and do some testing as to its capabilities. & Without inference &   \\ \hline
19.03.2021 & Add the sequential goals as another level in the hierarchy. & Still no inference. & \\ \hline
02.04.2021 & Run some exploratory simulations & How well does the model behave? Are there bad cases? & \\ \hline
09.04.2021 & Talk a  bout possible predictions that can be made with the model. & This includes a meeting. & \\ \hline
--- & Run final simulations and write paper. & & \\ \hline

\end{tabularx}


The previous time table was interrupted by the possibility of the new project. Now that this new project has taken shape, this new time table applies:

\begin{tabularx}{\textwidth}{
  l|
  >{\hsize=0.4\hsize\linewidth=\hsize}X|
  >{\hsize=0.2\hsize\linewidth=\hsize}X|
  >{\hsize=0.2\hsize\linewidth=\hsize}X}
End Date & Result & A priori comments & A posteriori comments \\ \hline \hline
30.04.2021 & Lit. research on savings and deadaptation & Are there many with contextual cues? & I found a lot of evidence. Moving on.\\ \hline
07.05.2021 & Finding model pars to explain observations & This is mostly already done &  \\\hline
14.05.2021 & Looking into error-clamp behavior & Maybe there's something to it. If there is, an extra week should be allocated to simulations. &  \\\hline
--- & Start writing the paper. & & \\ \hline

\end{tabularx}
\section{Results}

\begin{enumerate}
\item Quick de-adaptation/savings: gradual evidence accumulation during context inference will make the switching between internal models not be immediate.
\item Learning: The speed in which the internal models are adjusted given errors depends on how certain the inference is. More uncertain context inference will lead to slower learning.
\item Action selection: Action selection is sampled from the existing (relevant) internal models, weighted by their posterior probability during this trial.
\item Cue reliability adjustments: As more bad-cue trials are observed, the agent will lower the estimated reliability of the cue.
\item Error-clamp tappering off: Hundreds of error-clamp trials make adaptation seem to slowly disappear, but it never fully goes away.
\item Error-clamp lag: For a while, the tappering off does not begin, depending on how clear it is that error-clamp trials have started.
\end{enumerate}



\end{document}